%!TEX TS-program = xelatex

\useoutertheme{infolines}

\input {commonpackages.tex}

%\setmainfont {NanumMyeongjo}
\setsansfont {Noto Sans CJK KR}
\setmainfont {Noto Sans CJK KR}
\setmonofont[Scale=0.8]{DejaVu Sans Mono}

\input {lststyles.tex}

\hypersetup {
  colorlinks, linkcolor=blue
}

\title {Hideroot}
\input {beamauthor.tex}

\AtBeginSection[]
{
  \begin{frame}
    \frametitle{Index}
    \tableofcontents[currentsection]
  \end{frame}
}

\begin {document}

\begin{frame}
  \titlepage
\end{frame}

\section[Section]{루팅?}
\begin{frame}
  \frametitle{0x00. 루팅?}
  \framesubtitle{https://g.co/ABH}

  \begin{center}
    \includegraphics [width=100mm]{img/corrupted_nexus.png}
  \end{center}
\end{frame}

\begin{frame}
  \frametitle{0x00. 루팅?}
  \framesubtitle{https://g.co/ABH}

  \begin{itemize}
  \item 안드로이드 기기의 UID=0 권한을 가질 수 있게 도와주는 root helper 를 설치하는 행위
  \item<2-> 이를 통해, 앱이 원래대로라면 가질 수 없는 UID=0 권한을 획득할 수 있음
  \item<3-> 하지만 잠재적인 보안 이슈를 이유로, 많은 금융 앱들이 루트된 기기에서 동작을 거부함
  \item<4-> \textbf{이를 동작 가능하게 해 보자}
  \end{itemize}
\end{frame}
 
\section[Section]{루팅 탐지와 무력화}
\begin{frame}
  \frametitle{0x01. 루팅 탐지}
  \framesubtitle{librandomseccorp.so}

  \begin{itemize}
  \item 1세대: 실행 후 에러 코드 체크
  \item 2세대: /system/xbin/su 유무 체크 (I wonder why they didn't used PATH envar)
  \item 3세대: NDK 에서 libc 함수들을 이용해 체크
  \item 4세대: NDK 에서 system call 을 직접 호출해 체크 (iOS 에서 관찰됨)
  \end{itemize}
\end{frame}

\begin{frame}
  \frametitle{0x02. 루팅 탐지 무력화}
  \framesubtitle{Three Rings for the Elven-kings under the sky}

  \begin{itemize}
  \item 1세대: 가장 초보적 - 권한 허용 거부를 누른다.
  \item 2세대: XPosed 와 같은 Java hooking library
  \item 3세대: Cydia substrate (for NDK hook)
  \item 4세대: LKM rootkit, kernel patching, etc.
  \item 혹은 가장 귀찮은 방법: 앱을 패치해서 탐지 루틴을 무력화한다.
  \end{itemize}
\end{frame}

\section[Section]{Rootkit}
\begin{frame}
  \frametitle{0x03. User-mode rootkit - Xposed framework}
  \framesubtitle{Seven for the Dwarf-lords in their halls of stone}

  \begin{itemize}
  \item 일반 앱들에게, DalvikVM(or ART) 에 코드를 인젝션 할 수 있는 권한을 주는 프레임워크
  \item 이를 이용, 리플렉션이라는 흑법을 통해 순정 펌웨어(?) 를 커스텀 롬처럼 만들어주는 확장을 개발할 수 있게 된다
  \item 은행 앱에 훅을 주입하는것으로 2세대 루팅 탐지 우회 가능
  \end{itemize}
\end{frame}

\begin{frame}
  \frametitle{0x04. User-mode rootkit - Cydia Substrate}
  \framesubtitle{Nine for Mortal Men doomed to die}

  \begin{itemize}
  \item 네이티브 라이브러리 훅을 제공하는 프레임워크
  \item Apple iOS 의 Cydia Substrate 의 안드로이드 포트
  \item 이를 이용하면, NDK 의 libc 콜들을 후킹 가능
  \item 은행 앱이 부르는 libc 함수들을 후킹해서, 3세대 탐지 우회 가능
  \end{itemize}
\end{frame}

\begin{frame}
  \frametitle{0x05. Kernel mode rootkit}
  \framesubtitle{One for the Dark Lord on his dark throne}

  \begin{itemize}
  \item 인-커널 시스템 콜 후킹
  \item 커널 내에서 
  \end{itemize}
\end{frame}

\begin{frame}
  \frametitle{0x01. LKM rootkit}
  \framesubtitle{One Ring to rule them all}

  \begin{itemize}
  \item LKM
  \end{itemize}
\end{frame}

\section[Section]{끝}
\begin{frame}
  \frametitle{0x0D. 끝}
  \framesubtitle{FIN}

  \begin{center}
    Q \& A
  \end{center}
\end{frame}

\begin{frame}
  \frametitle{0x0B. License}
  \framesubtitle{}
  Copyright (C)  2016 perillamint\linebreak
  Permission is granted to copy, distribute and/or modify this document
  under the terms of the GNU Free Documentation License, Version 1.3
  or any later version published by the Free Software Foundation;\linebreak
  with no Invariant Sections, no Front-Cover Texts, and no Back-Cover Texts.
  A copy of the license is included in the section entitled "GNU
  Free Documentation License".
  \linebreak
  \linebreak
  %Repository address:\linebreak
  %\url{https://github.com/perillamint/K-HomeWRT-hack}
  \linebreak
  \linebreak
  \includegraphics [width=30mm]{img/gfdl-logo-small.png}
\end{frame}

\end {document}
